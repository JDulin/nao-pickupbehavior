\documentclass[12pt]{report}

\begin{document}

\title{Humanoid Robotics NAO Project}
\author{John Dulin}

\chapter{Weekly Goals}

\indent
This is an outline of the goals I intend to meet for the last 5 weeks of my Summer At The Edge internship, working on the
Humanoid Robotics with NAO Project.  Because of the lack of progress made in the first five weeks, these goals are definitive and
ambitious compared to the last five, but are still possible.  If solutions to any of these challenges are not successfully implemented 
by the end of the project, then a technical paper outlining the problems encountered, the methods I used to try to solve the problems, 
and specific models of other possible solutions will be made for any future work by TecEdge students on the NAO.

\chapter{Dates and Goals}

\section{July 20 - Finding, Tracking, and Grasping}

\indent
This will be a relatively simple behavior that uses the ALRedBallTracker to find a 3.5 centimeter red ball
(transformed from the 6 centimeter projection that ALRedBallDetection assumes) and approach it and pick it up using
the ALMotion library, specifically using the Whole Body generalized inverse kinematics solver to pick the ball up.

This will assume a stationary red ball and that the robot will not have to move to search for the ball.
The ball will be in plain view, elevated above the ground, somewhere within 5 feet of the robot and the range of its head's scanning motion. 
 So, the robot will scan an area with its head looking for the ball, turning and approaching the ball once one is detected. Then, it will activate
Whole Body motion to smoothly and safely pick it up.  Right now Whole Body motion cannot have NAO's hands reach the ground, although it's possible
with a custom module.  That is not our priority though at the moment.


\section{July 27 - Searching}


\indent
By July 27 I intend to have built a method for NAO to search an open area for a red ball not immediately in its field of view.  This will build on the head
scanning motion and pickup method built last week.  By finding ideal rates of walking and side to side head turning, NAO should be able to 
quickly scan an open space.  By default, the behavior will assume a rectangular or square space to search (at least at first) whose size can be 
defined as a program parameter.  Knowing the space size, I should be able to define walking paths for NAO that optimize his hunt, although the
rate of the search will be limited by how well ALRedBallTracker can identify balls while walking and turning the head.  Once a ball is spotted,
NAO will just go ahead and follow the algorithm built last week.  Searching in more complex, irregular environments can be done by using the sonar 
sensors and tracing NAO's position in absolute space  more rigorously, but that is not our priority and would be cut short by the sonar's voracious thirst
for power from the Li-Ion battery.


\section{August 3 - Tools}

\indent
With some basic functionality built, we have two stated goals for the project: make the challenge we are attacking more complex and meaningful by 
teaching NAO to collect a set of several balls in space, and then challenging NAO to manipulate a tool to make its job easier, like a wagon
or basket that holds what it finds.  I think we should focus on getting NAO to use a container, because it is a more challenging
problem, and one specifically for humanoids.  The wagon would be the most impressive tool to use because NAO would always have to think about where it
was and navigate with it, but it may not be possible to use in this time frame.  NAO's walking motion precariously balances him with the help of his 
arm motion and to have a V3.3 NAO use one of its arms to pull a wagon while walking would probably require a whole new walking method, in addition 
to the computer vision algorithms needed to teach NAO about the wagon and its handle.  So this week I would focus on determining if NAO pulling the wagon
was practical, and then build functionality for either pulling a wagon or carrying a basket.  The basket would probably have to be custom made:
light, strong enough to hold several ping-pong balls, and with a firm handle that is easily distinguishable.  Maybe even out of something like construction
paper.  This may be extra hard because the NAO's right hand only has 2 working fingers.

This problem would involve building a computer vision module specifically to see the handle of the container, modifying the pickup algorithm to
consider the container before and after grabbing a ball, and thinking about the location of the wagon in space.  This day I would like to show
Dr. Lee what approach I will be taking to the problem, why, and have conducted some simple tests on my computer vision approach to talk about. 


\section{August 10 - Look What I Can Do!}

\indent
The goal for August 10th is to have a NAO robot that can autonomously search a space for a single red ball, pick it up and place it in a container
(wagon or basket), and say "Look what I can do!"  This will require the robot to recognize the handle of the container several times (picking it up
at the start of the behavior and possibly after it picks up a ball and sets it down.), pick it up, and then execute all of its other functions properly.
This will be hard because as the program is worked on more and more complications to the original behavior may arise from throwing the container into the
mix.  If the program does not work, then I will show Dr. Lee a full analysis of the problems I faced in my work and what others I predict, along with
a set of possible technical solutions to each using the software tools at my disposal.  But hopefully that isn't necessary.


\section{August 17 - Project Review and Bonus Points:  Dealing with Unexpected Events or a Full Set of Balls.}

\indent
Dr. Williams from the Discovery Lab would like all summer projects to present a poster, a 2 minute 57 second video, a 12 page technical paper, and
a 37 slide presentation on their summer work at the August 14th project Open House.  If the project has met its milestones and I still have the robot, 
then in the meantime I will work on adding functionality for collecting a set of 3-5 balls instead of just one, or, if that is not practical, polishing the
program I already have by attempting to fix any bugs it has and accounting for unexpected events such as a dropped ball or handle, a moving ball,
or an arbitrary object that NAO has to learn to pickup.  Either way, I will compile everything I have learned over the summer into the presentation
forms asked for by Dr. Williams and a more concise version for Dr. Lee, so we can talk about the project's successes and failures.


\end{document}
